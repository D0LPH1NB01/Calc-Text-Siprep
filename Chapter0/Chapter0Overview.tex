\textbf{\underline{\large{Chapter 0 Overview: Reminders from Precalculus}}} \par

Analytic geometry and calculus are closely related subjects. Analytic geometry is the study of functions and relations as to how their graphs on the Cartesian System relate to the algebra of their equations. Traditionally, calculus has been the study of functions with a particular interest in tangent lines, maximum/minimum points, and area under the curve. (The Reform Calculus Movement places an emphasis on how functions change, rather than the static graph, and consider calculus to be a study of change and of motion.) Consequently, there is a great deal of overlap between the subjects. The advent of graphing calculators has blurred the distinctions between these fields and made subjects that had previously been strictly calculus topics easily accessible at the lower level. The point of this course is to thoroughly discuss the subjects of Analytic geometry that directly pertain to entry-level calculus and to introduce the concepts and algebraic processes of first semester calculus. Because we are considering these topics on the Cartesian Coordinate System, it is assumed that we are using Real Numbers, unless the directions state otherwise.

Last year’s text was designed to study the various families of functions in light of the main characteristics--or TRAITS--that the graphs of each family possess. Each chapter a different family and a) review what is known about that family from Algebra 2, b) investigate the analytic traits, c) introduce the calculus Rule that most applies to that family, d) put it all together in full sketches, and e) take one step beyond. In Chapter 0 of this text, we will review the traits that do not involve the derivative before going into the material that is properly part of calculus. Part of what will make the calculus much easier is if we have an arsenal of basic facts that we can bring to any problem. Many of these facts concern the graphs that we learned about individually in Precalculus. We need to synthesize all of this material into a cohesive body of information so that we can make the best use of it in calculus.

% possibly more readable version? %

\pagebreak
\parindent=0pt Analytic geometry and calculus are closely related subjects. \\
\setlength{\parindent}{24pt}    Analytic geometry: the study of functions and relations as to how their graphs on the Cartesian System relate to the algebra of their equations
    Calculus:
        (traditional def.) the study of functions with a particular interest in tangent lines, maximum/minimum points, and area under the curve.
        (The Reform Calculus Movement def.) ...places an emphasis on how functions change, rather than the static graph, and consider calculus to be a study of change and of motion.
Consequently, there is a great deal of overlap between the subjects. The advent of graphing calculators has blurred the distinctions between these fields and made subjects that had previously been strictly calculus topics easily accessible at the lower level. 

The point of this course is to
    1. thoroughly discuss the subjects of Analytic geometry that directly pertain to entry-level calculus
    2. introduce the concepts and algebraic processes of first semester calculus
Because we are considering these topics on the Cartesian Coordinate System, it is assumed that we are using Real Numbers, unless the directions state otherwise.

Last year’s text was designed to study the various families of functions in light of the main characteristics--or TRAITS-- that the graphs of each family possess.
Each chapter had a different family and
    a) review what is known about that family from Algebra 2
    b) investigate the analytic traits
    c) introduce the calculus rule that most applies to that family
    d) put it all together in full sketches, and\dots
    e) take one step beyond.

In Chapter 0 of this text, we will review the traits that do not involve the derivative before going into the material that is properly part of calculus. 

***Part of what will make the calculus much easier is if we have an arsenal of basic facts that we can bring to any problem.
    - Many of these facts concern the graphs that we learned about individually in Precalculus.
    - We need to synthesize all of this material into a cohesive body of information so that we can make the best use of it in calculus.