\textbf{\underline{\large{Chapter 0 Overview: Reminders from Precalculus}}} \par

Analytic geometry and calculus are closely related subjects. As a reminder: \par

\begin{tcolorbox}[definition]
    \begin{tabbing}
        \textit{Analytic Geometry} $\rightarrow$ \= Definition: The study of functions and relations as to how their \\ 
        \> graphs on the Cartesian system relate to the algebra of their \\
        \> equations. \\[5.5pt]
        \textit{Calculus} $\rightarrow$ \= Definition (traditional): The study of functions with a particular interest \\
        \> in tangent lines, maximum/minimum points, and area under the curve.
    \end{tabbing}
\end{tcolorbox}

The Reform Calculus Movement places an emphasis on how functions change, rather than the static graph, and consider calculus to be a study of change and of motion.

The point of this textbook is to 1) thoroughly discuss the subjects of analytic geometry that directly pertain to entry-level calculus and 2) introduce the concepts and algebraic processes of first semester calculus. Because we are considering these topics on the Cartesian coordinate system, it is assumed that we are using real numbers, unless the directions state otherwise.

Last year's text was designed to study the various families of functions in light of the main characteristics---or \textbf{traits}---that the graphs of each family possess.
In this text, we will revisit these families of functions and \begin{enumerate}
    \item Review what is known about that family from Algebra 2.
    \item Investigate the analytic traits.
    \item Introduce the calculus rule that most applies to that family.
    \item Put it all together in full sketches.
    \item Take one step beyond.
\end{enumerate}

In Chapter 0 of this text, we will review the traits that do not involve the derivative before going into the material that is properly part of calculus. Part of what will make the calculus much easier is if we have an arsenal of basic facts that we can bring to any problem. Many of these facts concern the graphs that we learned about individually in precalculus. We need to synthesize all of this material into a cohesive body of information so that we can make the best use of it in calculus.
