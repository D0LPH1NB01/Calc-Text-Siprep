\textbf{\underline{\large{1.3: Trig, Trig Inverse, and Log Rules}}} \par

Remember:
\begin{center}
    \fbox{\fbox{\begin{minipage}{0.96\textwidth}
        \vspace{11pt}
        \begin{align*}
            & \text{The Product Rule: } f'(x) = U \cdot \dfrac{dV}{dx} + V \cdot \dfrac{dU}{dx} \\[11pt]
            & \text{The Quotient Rule: } f'(x) = \dfrac{V \cdot \dfrac{dU}{dx} - U \cdot \dfrac{dV}{dx}}{V^2} \\[11pt]
        \end{align*}
    \end{minipage}}}
\end{center}

\begin{tcolorbox}[objective]
    \begin{center}
        OBJECTIVES \\[11pt]
    \end{center}
    Find the Derivative of a Product or Quotient of Two Functions. 
\end{tcolorbox}

\bigskip

\hypertarget{Product Rule}{\textbf{\large{The Product Rule}}} \par

\begin{tcolorbox}[example]
    \textbf{Ex 1.3.1: } $\diff \left[x^2\sin (x)\right]$ 
\end{tcolorbox}
\begin{tcolorbox}[solution]
    \textbf{Sol 1.3.1: } \begin{align*}
        \diff \left[x^2\sin (x)\right] & = x^2 \cdot \cos (x) + \sin (x) \cdot (2x) \\[11pt]
        & = \boxed{x^2\cos (x) + 2(x)\sin (x)}
    \end{align*} 
\end{tcolorbox} \vspace{11pt}

\begin{tcolorbox}[example]
    \textbf{Ex 1.3.2: } $\diff \left[5^x \cos (x)\right]$
\end{tcolorbox}
\begin{tcolorbox}[solution]
    \textbf{Sol 1.3.2: } \begin{align*}
        \diff \left[5^x \cos (x)\right] & = 5^x \cdot (-\sin (x)) + \cos (x) \cdot (5^x \ln 5) \\[11pt]
        & = \boxed{5^x \left(\ln (5)\cos (x) + \sin(x)\right)}
    \end{align*}
\end{tcolorbox}

The product rule is pretty straightforward. The tricky part is simplifying the algebra. \par

\begin{tcolorbox}[example]
    \textbf{Ex 1.3.3: } If $f(x) = x^2e^{-\frac{x}{2}}$, find $f'(x)$ 
\end{tcolorbox}
\begin{tcolorbox}[solution]
    \textbf{Sol 1.3.3: } \begin{align*}
        & U = x^2, \; \dfrac{dU}{dx} = 2x \\[11pt]
        & V = e^{-\frac{x}{2}}, \; \dfrac{dV}{dx} = e^{-\frac{x}{2}} \cdot \left(-\dfrac{1}{2}\right) = -\dfrac{1}{2}e^{-\frac{x}{2}} \\[11pt]
        & f'(x) \begin{aligned}[t]
            & = x^2 \cdot \left(-\dfrac{1}{2}e^{-\frac{x}{2}}\right) + e^{-\frac{x}{2}} \cdot 2x \\[11pt]
            & = \boxed{xe^{-\frac{x}{2}}\left(-\dfrac{1}{2}x + 2\right)}
        \end{aligned}
    \end{align*}
\end{tcolorbox} \vspace{11pt}

\begin{tcolorbox}[example]
    \textbf{Ex 1.3.4: } $\diff \left[x\sqrt{1 - x^2}\right]$ 
\end{tcolorbox}
\begin{tcolorbox}[solution]
    \textbf{Sol 1.3.4: } \begin{align*}
        & U = x, \; \dfrac{dU}{dx} = 1 \\[11pt]
        & V = \sqrt{1 - x^2} = \left(1 - x^2\right)^\frac{1}{2}, \; \dfrac{dV}{dx} = \dfrac{1}{2}\left(1 - x^2\right)^{-\frac{1}{2}} \cdot (-2x) = -\dfrac{x}{\sqrt{1 - x^2}} \\[11pt]
        & \diff \left[x\sqrt{1 - x^2}\right] \begin{aligned}[t]
            & = x \cdot \left(-\dfrac{x}{\sqrt{1 - x^2}}\right) + \sqrt{1 - x^2} \cdot 1 \\[11pt]
            & = \dfrac{-x^2 + (1 - x^2)}{\sqrt{1 - x^2}} \\[11pt]
            & = \boxed{\dfrac{1 - 2x^2}{\sqrt{1 - x^2}}}
        \end{aligned}
    \end{align*}
\end{tcolorbox} \vspace{11pt}

\begin{tcolorbox}[example]
    \textbf{Ex 1.3.5: } $\diff \left[(2x - 3)^8\left(3x^2 - 1\right)^7\right]$ 
\end{tcolorbox}
\begin{tcolorbox}[solution]
    \textbf{Sol 1.3.5: } \begin{align*}
        & U = (2x - 3)^8, \; \dfrac{dU}{dx} = 8(2x - 3)^7 \cdot 2 = 16(2x - 3)^7 \\[11pt]
        & V = \left(3x^2 - 1\right)^7, \; \dfrac{dV}{dx} = 7\left(3x^2 - 1\right)^6 \cdot 6x  = 42x\left(3x^2 - 1\right)^6\\[11pt]
        & \diff \left[(2x - 3)^8\left(3x^2 - 1\right)^7\right] = (2x - 3)^8 \cdot 42x\left(3x^2 - 1\right)^6 + \left(3x^2 - 1\right)^7 \cdot 16(2x - 3)^7 \\[11pt]
        & \text{This, then, is factorable.} \\[11pt]
        & \diff \left[(2x - 3)^8\left(3x^2 - 1\right)^7\right] \begin{aligned}[t]
            & = 42x(2x - 3)^8\left(3x^2 - 1\right)^6 + 16\left(3x^2 - 1\right)^7 16(2x - 3)^7 \\[11pt]
            & = 2(2x - 3)^7\left(3x^2 - 1\right)^6\left(21x(2x - 3) + 8\left(3x^2 - 1\right)\right) \\[11pt]
            & = 2(2x - 3)^7\left(3x^2 - 1\right)^6\left(42x^2 - 63x + 24x^2 - 8\right) \\[11pt]
            & = \boxed{2(2x - 3)^7\left(3x^2 - 1\right)^6\left(66x^2 - 63x - 8)\right)}
        \end{aligned}
    \end{align*}
\end{tcolorbox}

Remember that in \hyperlink{Section 1.1}{Section 1.1} we said that we would need the Product Rule to deal with the derivative of a function where the variable is in both the base and the exponent. We can now address that situation. \par

\begin{tcolorbox}[example]
    \textbf{Ex 1.3.6: } $\diff \left[(\cos (x))^{x^2}\right]$
\end{tcolorbox}
\begin{tcolorbox}[solution]
    \textbf{Sol 1.3.6: } \begin{align*}
        \diff \left[(\cos (x))^{x^2}\right] & = \diff \left[e^{x^2\ln (\cos (x))}\right] \\[11pt]
        & = e^{x^2\ln (\cos (x))} \cdot \left(x^2 \cdot \dfrac{1}{\cos (x)} \cdot -(\sin (x)) + \ln (\cos (x)) \cdot 2x\right) \\[11pt]
        & = \boxed{(\cos (x))^{x^2}\left(2x\ln (\cos (x)) - x^2\tan (x)\right)}
    \end{align*} 
\end{tcolorbox}

\bigskip

\textbf{\large{The Quotient Rule}} \par

\begin{tcolorbox}[example]
    \textbf{Ex 1.3.7: } $\diff \left[\dfrac{6x}{x^2 + 4}\right]$
\end{tcolorbox}
\begin{tcolorbox}[solution]
    \textbf{Sol 1.3.7: } \begin{align*}
        & U = 6x, \; \dfrac{dU}{dx} = 6 \\[11pt]
        & V = x^2 + 4, \; \dfrac{dV}{dx} = 2x \\[11pt]
        & \diff \left[\dfrac{6x}{x^2 + 4}\right] \begin{aligned}[t]
            & = \dfrac{\left(x^2 + 4\right) \cdot 6 - 6x \cdot 2x}{\left(x^2 + 4\right)^2} \\[11pt]
            & = \dfrac{6x^2 + 24 - 12x^2}{\left(x^2 + 4\right)} \\[11pt]
            & = \boxed{\dfrac{24 - 6x^2}{\left(x^2 + 4\right)}}
        \end{aligned}
    \end{align*}
\end{tcolorbox} \vspace{11pt}    

\begin{tcolorbox}[example]
    \textbf{Ex 1.3.8: } $\diff \left[\dfrac{x^2 + 2x - 3}{x - 4}\right]$ 
\end{tcolorbox}
\begin{tcolorbox}[solution]
    \textbf{Sol 1.3.8: } \begin{align*}
        & U = x^2 + 2x - 3, \; \dfrac{dU}{dx} = 2x + 2 \\[11pt]
        & V = x - 4, \; \dfrac{dV}{dx} = 1 \\[11pt]
        & \diff \left[\dfrac{x^2 + 2x - 3}{x - 4}\right] \begin{aligned}[t]
            & = \dfrac{(x - 4) \cdot (2x + 2) - \left(x^2 + 2x - 3\right) \cdot 1}{(x - 4)^2} \\[11pt]
            & = \dfrac{2x^2 - 6x - 8 - x^2 - 2x + 3}{(x - 4)^2} \\[11pt]
            & = \boxed{\dfrac{x^2 - 8x - 5}{(x - 4)^2}}
        \end{aligned}
    \end{align*} 
\end{tcolorbox} \vspace{11pt}

\begin{tcolorbox}[example]
    \textbf{Ex 1.3.9: } $\diff \left[\dfrac{x^2 - 4x + 3}{2x^2 - 5x - 3}\right]$ 
\end{tcolorbox}
\begin{tcolorbox}[solution]
    \textbf{Sol 1.3.9: } Notice that this problem becomes much easier if we simplify before applying the Quotient Rule. \begin{align*}
        \diff \left[\dfrac{x^2 - 4x + 3}{2x^2 - 5x - 3}\right] & = \diff \left[\dfrac{(x - 1)(x - 3)}{(2x + 1)(x - 3)}\right] \\[11pt] 
        & = \diff \left[\dfrac{x - 1}{2x + 1}\right] \\[11pt]
        & U = x - 1, \; \dfrac{dU}{dx} = 1 \\[11pt]
        & V = 2x + 1, \; \dfrac{dV}{dx} = 2 \\[11pt]
        & \diff \left[\dfrac{x - 1}{2x + 1}\right] \begin{aligned}[t]
            & = \dfrac{(2x + 1) \cdot 1 - (x - 1) \cdot 2}{(2x + 1)^2} \\[11pt]
            & = \boxed{\dfrac{3}{(2x + 1)^2}}
        \end{aligned}
    \end{align*}
\end{tcolorbox} \vspace{11pt}

\begin{tcolorbox}[example]
    \textbf{Ex 1.3.10: } $\diff \left[\dfrac{\cot (3x)}{x^2 + 1}\right]$ 
\end{tcolorbox}
\begin{tcolorbox}[solution]
    \textbf{Sol 1.3.10: } \begin{align*}
        & U = \cot (3x), \; \dfrac{dU}{dx} = -\csc^2 (3x) \cdot 3 = -3\csc^2 (3x)\\[11pt]
        & V = x^2 + 1, \; \dfrac{dV}{dx} = 2x \\[11pt]
        & \diff \left[\dfrac{\cot (3x)}{x^2 + 1}\right] \begin{aligned}[t]
            & = \dfrac{\left(x^2 + 1\right) \cdot \left(-3\csc^2 (3x)\right) - \cot (3x) \cdot 2x}{\left(x^2 + 1\right)^2} \\[11pt]
            & = \dfrac{-3x^2\csc^2 (3x) - 3\csc^2 (3x) - 2x\cot (3x)}{\left(x^2 + 1\right)^2} \\[11pt]
            & = \boxed{-\dfrac{\csc^{2}(3x)\left(3x^2 + 3\right) + 2x \cot(3x)}{\left(x^2 + 1\right)^2}}
        \end{aligned}
    \end{align*}
\end{tcolorbox} 

As with the Product Rule, the difficulty with the Quotient Rule arises from the algebra needed to simplify our answer. \par

\begin{tcolorbox}[example]
    \textbf{Ex 1.3.11: } If $y = \dfrac{4x}{\sqrt{x^2 + 4}}$, find $\deriv$ 
\end{tcolorbox}
\begin{tcolorbox}[solution]
    \textbf{Sol 1.3.11: } \begin{align*}
        & U = 4x, \; \dfrac{dU}{dx} = 4 \\[11pt]
        & V = \sqrt{x^2 + 4} = \left(x^2 + 4\right)^\frac{1}{2}, \; \dfrac{dV}{dx} = \dfrac{1}{2}\left(x^2 + 4\right)^{-\frac{1}{2}} \cdot 2x = \dfrac{2x}{2\sqrt{x^2 + 4}} \\[11pt]
        & \deriv \begin{aligned}[t]
            & = \dfrac{\sqrt{x^2 + 4} \cdot 4 - 4x \cdot \dfrac{2x}{2\sqrt{x^2 + 4}}}{x^2 + 4} \\[11pt]
            & = \dfrac{\dfrac{4\left(x^2 + 4\right)}{\sqrt{x^2 + 4}} - \dfrac{4x^2}{\sqrt{x^2 + 4}}}{x^2 + 4} \\[11pt]
            & = \dfrac{4x^2 + 16 - 4x^2}{\left(x^2 + 4\right)^\frac{3}{2}} \\[11pt]
            & = \boxed{\dfrac{16}{\left(x^2 + 4\right)^\frac{3}{2}}}
        \end{aligned}
    \end{align*}
\end{tcolorbox} \vspace{11pt}

\begin{tcolorbox}[example]
    \textbf{Ex 1.3.12: } Find the equation of the tangent line to $f(x) = \dfrac{x}{\sqrt{x^2 + 9}}$ at $x = -\sqrt{7}$.
\end{tcolorbox}
\begin{tcolorbox}[solution]
    \textbf{Sol 1.3.12: } As we recall, for the equation of a line, we need a point and a slope. \begin{align*}
        \text{The point: } f\left(-\sqrt{7}\right) & = \dfrac{-\sqrt{7}}{\sqrt{\left(-\sqrt{7}\right)^2 + 9}} \\[11pt]
        & = -\dfrac{\sqrt{7}}{4} \rightarrow \left(-\sqrt{7}, -\dfrac{\sqrt{7}}{4}\right) 
    \end{align*} 
    The slope is the derivative at the given x-value: \begin{align*}
        & U = x, \; \dfrac{dU}{dx} = 1 \\[11pt]
        & V = \sqrt{x^2 + 9} = \left(x^2 + 9\right)^\frac{1}{2}, \; \dfrac{dV}{dx} = \dfrac{1}{2}\left(x^2 + 9\right)^{-\frac{1}{2}} \cdot 2x = \dfrac{2x}{2\sqrt{x^2 + 9}} \\[11pt]
        & \deriv = \dfrac{\sqrt{x^2 + 9} \cdot 1 - x \cdot \dfrac{2x}{2\sqrt{x^2 + 9}}}{x^2 + 9} 
    \end{align*} 
    Rather than simplify the algebra, we can find the slope by substituting $x = -\sqrt{7}$: \begin{align*}
        \deriv \eval_{x = -\sqrt{7}} & = \dfrac{\sqrt{\left(-\sqrt{7}\right)^2 + 9} \cdot 1 - \left(-\sqrt{7}\right) \cdot \dfrac{2\left(-\sqrt{7}\right)}{2\sqrt{\left(-\sqrt{7}\right)^2 + 9}}}{\left(-\sqrt{7}\right)^2 + 9} \\[11pt]
        & = \dfrac{4 - \dfrac{7}{4}}{16} \\[11pt]
        & = \dfrac{9}{64}
    \end{align*} 
    The tangent line equation is therefore: \begin{align*}
        \boxed{y + \dfrac{\sqrt{7}}{4} = \dfrac{9}{64}\left(x + \sqrt{7}\right)} 
    \end{align*} 
\end{tcolorbox}

\newpage

\textbf{\large{1.3 Free Response Homework Set A}} \par

Find the derivatives of the given functions. Simplify where possible. \par

\twoquestion{1. $y = t^3\cos (t)$}{2. $y = \left(2x - 5\right)^4\left(8x^2 - 5\right)^{-3}$} \\[11pt]
\twoquestion{3. $y = \dfrac{\tan (x) - 1}{\sec (x)}$}{4. $y = \dfrac{\sin (x)}{x^2}$} \\[11pt]
\twoquestion{5. $y = xe^{-x^2}$}{6. $y = \dfrac{r}{\sqrt{r^2 + 1}}$} \\[11pt]
\twoquestion{7. $y = e^{x\cos (x)}$}{8. $y = e^{-5x}\cos (3x)$} \\[11pt]
\twoquestion{9. $y = x\sin \left(\dfrac{1}{x}\right)$}{10. $y = \ln \left(e^{-x} + xe^{-x}\right)$} \\[11pt]
\twoquestion{11. $y = \dfrac{\sec^{-1} (x)}{x}$}{12. $y = \dfrac{\sin (x)}{x^2}$} \\[11pt]
\twoquestion{13. $y = \left(1 + x^2\right) \tan^{-1}(x)$}{14. $y = \ln \left(x^2 + 4\right) - x\tan^{-1}\left(\dfrac{x}{2}\right)$} \\[11pt]
\twoquestion{15. $f(x) = x\sqrt{\ln x}$}{16. $g(x) = (1 + 4x)^5\left(3 + x - x^2\right)^8$} \\[11pt]
\twoquestion{17. $f(x) = x\cos^{-1} (x) - \sqrt{1 - x^2}$}{18. $g(x) = \cos^{-1} (x) + x\sqrt{1 - x^2}$} \\[11pt]

Complete the following. \par

\twoquestion{19. $\diff \left[\dfrac{3x^2 + 4x - 3}{x^2 - 9}\right]$}{20. $\diff \left[\dfrac{x^3 - 2x^2 - 5x + 6}{x + 2}\right]$} \\[11pt]
\twoquestion{21. $\diff \left[\dfrac{x^5 - 12x^3 - 19x}{3x^3}\right]$}{22. $\diff \left[\dfrac{3x + 3}{x^3 + 1}\right]$} \\[11pt]
\twoquestion{23. $\diff \left[\dfrac{x - 4}{x^2 - 9x + 20}\right]$}{24. $\diff \left[\dfrac{\tan (x) + 5}{\sin (x)}\right]$} \\[11pt]
\twoquestion{25. $\diff \left[\dfrac{\sin (x)}{1 - \cos (x)}\right]$}{26. $\diff \left[\dfrac{x^2}{\cos (x)}\right]$} \\[11pt]
\twoquestion{27. $y = \dfrac{x^2 - 3}{x^2 - 4}$, find $\deriv$}{28. $f(x) = \dfrac{x^2 + 2x - 8}{x^2 - x - 3}$, find $f'(x)$} \\[11pt]
\twoquestion{29. $y = \dfrac{x^2 + 2x - 3}{x - 4}$, find $y'$}{30. $f(x) = \dfrac{x}{\ln x}$, find $f'(x)$} \\[11pt]
\twoquestion{31. $h(t) = \left(\dfrac{1 + x^2}{1 - x^2}\right)^{17}$, find $h'(t)$}{32. $y = \dfrac{\tan (x)}{\cos (x) - 3}$, find $\deriv$} \\[11pt]
\twoquestion{33. $f(x) = \left(x\sin (2x) + \tan^{4} \left(x^7\right)\right)^5$, find $f'(x)$}{34. $f(x) = e^x - x^2\arctan{x}$, find $f'(x)$} \\[11pt]
\twoquestion{35. $f(x) = \dfrac{\tan (x)}{\tan (x) + 1}$, find $f'\left(\dfrac{\pi}{4}\right)$}{36. $y = x^2\sqrt{5 - x^2}$, find $y'(1)$} \\[11pt]

\textbf{\large{1.3 Free Response Homework Set B}} \par

Complete the following. \par

\onequestion{1. Find the first derivative for the following function: $x(t) = e^{t^2}\sin \left(t^2 - 5t^4\right)$} \\[11pt]
\onequestion{2. Find the first derivative for the following function: $x(t) = e^{5t}\tan \left(3t^4\right)$} \\[11pt]
\onequestion{3. Find the first derivative for the following function: $y = \dfrac{x^2 + 2x - 15}{x - 3}$} \\[11pt]
\onequestion{4. Find the first derivative for the following function: $x(t) = e^t\left(t^2 - 5t^4\right)$} \\[11pt]
\twoquestion{5. $\diff \left[\dfrac{e^x + 7x^2 + 5}{\sin \left(x^3\right)}\right]$}{6. $\diff \left[e^{\sin(x)} \ln \left(\cot \left(e^x\right)\right)\right]$} \\[11pt]
\twoquestion{7. $\diff \left[x^2\sin \left(x^2\right) + \dfrac{x + 1}{\ln x}\right]$}{8. $\diff \left[x^2\cos \left(x^2\right) + \dfrac{e^x}{x}\right]$} \\[11pt]
\twoquestion{9. $\diff \left[x^5\ln (5x + 4) + \dfrac{x}{\ln x}\right]$}{10. $\diff \left[\dfrac{\cos \left(x^2 - 3\right)}{e^{-5x}}\right]$} \\[11pt]
\twoquestion{11. $\diff \left[e^{x^2}\cos (x)\right]$}{12. $\diff \left[\dfrac{1 + \tan (x)}{\ln (4x)}\right]$} \\[11pt]
\twoquestion{13. $\diff \left[\sin (t)\tan (t)\right]$}{14. $\diff \left[\dfrac{1 + \ln x}{\csc (x)}\right]$} \\[11pt]
\twoquestion{15. $\diff \left[e^{5x^4}\ln \left(\sin (x)\right)\right]$}{16. $\diff \left[5x\sin (x) + e^{2x} - \ln \left(3x^2 + 1\right) + \dfrac{x}{x^2 + 1}\right]$} \\[11pt]
\twoquestion{17. $\diff \tan \left(e^x\right)\left(x^4 - 5x^3 + x\right)$}{18. $\diff \left[\dfrac{5x + 2}{\ln (3x + 7)}\right]$} \\[11pt]
\twoquestion{19. $\diff \left[\dfrac{x^5 - 12c^3 - 19c}{3c^3}\right]$}{20. $\diff \left[\diff \left[\sin^2 (4x + 2)\right]\right]$} \\[11pt]
\twoquestion{21. $g(z) = \left(\dfrac{e^{5z}}{1 + \ln z}\right)^{118}$, find $g'(z)$}{22. $g(t) = \left(\dfrac{t^2 - 4}{1 - t^2}\right)^{15}$, find $g'(t)$} \\[11pt]
\twoquestion{23. $y = \tan^{-1} \left(\dfrac{2e^x}{1 - e^{2x}}\right)$, find $y'$}{24. $f(x) = x^2\arccos (x)$, find $f'(x)$} \\[11pt]
\twoquestion{25. $y = \ln \left(u^2 + 1\right) - u\cot^{-1} (u)$, find $\dfrac{dy}{du}$}{26. $y = \cos^{-1} \left(\dfrac{x - 1}{x + 1}\right)$} \\[11pt]
\twoquestion{27. $f(t) = c\sin^{-1} \left(\dfrac{t}{c}\right) - \sqrt{c^2 - t^2}$, find $f'(t)$}{28. $y = 4\sin^{-1} \left(\dfrac{1}{2}x\right) + x\sqrt{4 - x^2}$} \\[11pt]
\onequestion{29. If $h(1) = 5$ and $h'(1) = 3$, find $f'(1)$ if $f(x) = \left(h(x)\right)^4 + x \ln \left(h(x)\right)$} \\[11pt]

\textbf{\large{1.3 Multiple Choice Homework}} \par

\begin{questions}
    \question If $y = x^2\cos (2x)$, then $\deriv = $ \\

    \begin{oneparchoices}
        \choice $-2x\sin (2x)$
        \choice $-4x\sin (2x)$
        \choice $2x\left(\cos (2x) - \sin (2x)\right)$ \\[11pt]
        \makebox[0.17 \textwidth] \choice $2x\left(\cos (2x) - x\sin (2x)\right)$
        \makebox[0.2 \textwidth] \choice $2x\left(\cos (2x) + \sin (2x)\right)$
    \end{oneparchoices} \par \horizontalline

    \question If $x(t) = 2t\cos \left(t^2\right)$, find $x'(t)$. \\

    \begin{oneparchoices}
        \choice $x'(t) = \sin \left(t^2\right) + 3$
        \choice $x'(t) = -\sin \left(t^2\right) + 4$ 
        \choice $x'(t) = \sin \left(t^2\right) + 2$ \\[11pt]
        \makebox[0.14 \textwidth] \choice $x'(t) = -4t^2\sin \left(t^2\right)$ 
        \makebox[0.15 \textwidth] \choice $x'(t) = -4t^2\sin \left(t^2\right) + 2\cos \left(t^2\right)$ 
    \end{oneparchoices} \par \horizontalline

    \question If $f(x) = x\tan (x)$, then $f'\left(\dfrac{\pi}{4}\right) = $ \\

    \begin{oneparchoices}
        \choice $1 - \dfrac{\pi}{2}$ 
        \choice $1 + \dfrac{\pi}{2}$
        \choice $1 + \dfrac{\pi}{4}$
        \choice $1 - \dfrac{\pi}{4}$ 
        \choice $\dfrac{\pi}{2} - 1$
    \end{oneparchoices} \par \horizontalline

    \question If $f$ is a function that is differentiable throughout its domain and is defined by $f(x) = \dfrac{1 + e^x}{\sin \left(x^2\right)}$, then the value of $f'(0) = $ \\

    \begin{oneparchoices}
        \choice $-1$ 
        \choice $0$
        \choice $1$
        \choice $e$
        \choice nonexistent
    \end{oneparchoices} \par \horizontalline
    
    \question If $y = \dfrac{5x - 4}{4x - 5}$, then $\deriv = $ \\

    \begin{oneparchoices}
        \choice $-\dfrac{9}{(4x - 5)^2}$
        \choice $\dfrac{9}{(4x - 5)^2}$ 
        \choice $\dfrac{40x - 41}{(4x - 5)^2}$
        \choice $\dfrac{40x + 41}{(4x - 5)^2}$
        \choice $\dfrac{5}{4}$
    \end{oneparchoices} \par \horizontalline

    \question If $y = \dfrac{3 - 2x}{3x + 2}$, then $\deriv = $ \\

    \begin{oneparchoices}
        \choice $\dfrac{12x + 2}{(3x + 2)^2}$
        \choice $\dfrac{12x - 2}{(3x + 2)^2}$
        \choice $\dfrac{13}{(3x + 2)^2}$
        \choice $-\dfrac{13}{(3x + 2)^2}$
        \choice $-\dfrac{2}{3}$
    \end{oneparchoices} \par \horizontalline

    \question If $y = \dfrac{3}{4 + x^2}$, then $\deriv = $ \\

    \begin{oneparchoices}
        \choice $-\dfrac{6x}{\left(4 + x^2\right)^2}$
        \choice $\dfrac{3x}{\left(4 + x^2\right)^2}$
        \choice $\dfrac{6x}{\left(4 + x^2\right)^2}$
        \choice $-\dfrac{3x}{\left(4 + x^2\right)^2}$
        \choice $\dfrac{3}{2x}$
    \end{oneparchoices} \par \horizontalline

    \question Which of the following statements must be true? \begin{align*}
        & \text{I. } \diff \left[x\tan (x)\right] = x\tan (x) + x\sec^2 (x) \\[11pt]
        & \text{II. } \diff \left[\dfrac{3}{4 + x^2}\right] = \dfrac{-6x}{\left(4 + x^2\right)^2} \\[11pt]
        & \text{III. } \diff \left[\sqrt{1 - x}\right] = \dfrac{1}{2\sqrt{1 - x}}
    \end{align*}

    \begin{oneparchoices}
        \choice I only 
        \choice II only 
        \choice III only \\[11pt]
        \makebox[0.2 \textwidth] \choice I and II only 
        \makebox[0.25 \textwidth] \choice I, II, and III
    \end{oneparchoices} \par \horizontalline
\end{questions}