\textbf{\underline{\large{1.9: Logarithmic Differentiation}}} \par

With implicit differentiation and the Chain Rule, we learned some powerful tools for differentiating functions and relations. The Product and Quotient Rules also allowed us to take derivatives of certain functions that would otherwise be impossible to differentiate. Sometimes, however, with very complex functions, it becomes easier to manipulate an equation so that it is easier to take the derivative. This is where logarithmic differentiation comes in. \par

\begin{tcolorbox}[objective]
    \begin{center}
        OBJECTIVES \\[11pt]
    \end{center}
    Determine When It Is Appropriate to Use Logarithmic Differentiation. \\
    Use Logarithmic Differentiation to Take The Derivatives of Complicated Functions.
\end{tcolorbox}

Before we begin, it would be helpful to look at a few exponent and logarithm rules that we should recall from algebra and precalculus. 

\begin{center}
    \fbox{\fbox{\begin{minipage}{0.96\textwidth}
        \begin{align*}
            & a^xa^y = a^{x + y} && \log_a{x} + \log_a{y} = \log_a{xy} \\[11pt]
            & \dfrac{a^x}{a^y} = a^{x - y} && \log_a{x} - \log_a{y} = \log_a{\dfrac{x}{y}} \\[11pt]
            & \left(a^x\right)^y = a^{xy} && \log_a{x^n} = n\log_a{x} \\
        \end{align*}
    \end{minipage}}}
\end{center}