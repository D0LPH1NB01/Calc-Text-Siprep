\textbf{\underline{\large{1.1: The Power and Exponential Rules with the Chain Rule}}} \par

In Precalculus we developed the idea of the derivative geometrically. That is, the derivative initially arose from our need to find the slope of the tangent line. In Chapter 2 and 3, that meaning, its link to limits, and other conceptualizations of the derivative will be explored. In this chapter, we are primarily interested in how to find the derivative and what it is used for. \par

\begin{tabbing}
    \textit{Derivative}: \= $\rightarrow \text{Definition: } f(x) = \lim_{h \to 0} \dfrac{f(x + h) - f(x)}{h}$ \\[5.5pt]
    \> $\rightarrow$ {Means:  The function that yields the slope of the tangent line.}
\end{tabbing}
\begin{tabbing}
    \textit{Numerical Derivative}: \= $\rightarrow \text{Definition: } f'(a) = \lim_{x \to a} \dfrac{f(x) - f(a)}{x - a}$ \\[5.5pt]
    \> $\rightarrow$ Means: \= The numerical value of the slope of the tangent line at \\
    \> \> $x = a$ 
\end{tabbing}

\fbox{\fbox{\begin{minipage}{0.96 \textwidth}
    \vspace{11pt}
    \begin{center}
        Symbols for the Derivative
        \\[11pt]
        $\hfill \dfrac{dy}{dx} = \text{"d - y - d - x"} \hfill f'(x) = \text{"f prime of x"} \hfill y' = \text{"y prime"} \hfill$ \\[11pt] 
        $\hfill \diff = \text{"d - d - x}" \hfill D_{x} = \text{"d sub x"} \hfill$
    \end{center} 
    \vspace{11pt}
\end{minipage}}}

\fbox{\begin{minipage}{0.96 \textwidth}
    \begin{center}
        OBJECTIVES \\[11pt]
    \end{center}
    Use the Power Rule and Exponential Rules to find Derivatives. \\
    Find the Derivative of Composite Functions. 
\end{minipage}}

\textbf{Key Idea from PreCalculus:} The derivative yields the slope of the tangent line. (But there is more to it than that). \par
The first and most basic derivative rule is the Power Rule. Among the last rules we learned in PreCalculus were the Exponential Rules. They look similar to one another, therefore it would be a good idea to view them together. 

\fbox{\fbox{\begin{minipage}{0.96 \textwidth}
    \begin{align*}
        & \text{The Power Rule:} && \text{The Exponential Rules:} \\[11pt]
        & \diff \left[x^n\right] = nx^{n - 1} && \diff \left[e^x\right] = e^x \\[11pt]
        & && \diff \left[a^x\right] = a^x \cdot \ln{a} \\
    \end{align*}
\end{minipage}}}

The difference between these is where the variable is. The Power Rule applies when the variable is in the \textit{base}, while the Exponential Rules apply when the variable is in the \textit{exponent}. The difference between the two Exponential rules is what the base is. $e = 2.718281828459 \dots$, while $a$ is any positive number other than $1$. \par 


\textbf{Ex 1.1.1:} Find a) $\diff\left[x^5\right]$ and b) $\diff\left[5^x\right]$ \\[11pt]
\textbf{Sol 1.1.1:} The first is a case of the Power Rule while the second is a case of the second Exponential Rule. Therefore, 
\begin{center}
$\hfill \text{a) } \diff\left[x^5\right] = \boxed{5x^4} \hfill \text{b) } \diff\left[5^x\right] = \boxed{5^x \ln{5}} \hfill$
\end{center}  

There are a few other basic rules that we need to remember. 

\fbox{\fbox{\begin{minipage}{0.96 \textwidth}
    \begin{align*}
        & \diff[\text{constant}] = 0 \\[11pt]
        & \diff[cx^n] = (cn)x^{n - 1} \\[11pt]
        & \diff[f(x) + g(x)] = \diff[f(x)] + \diff[g(x)] \\
    \end{align*}
\end{minipage}}}

These rules allow us to easily differentiate a polynomial term by term.

\textbf{Ex 1.1.2:} $y = 3x^2 + 5x + 1$; find $\dfrac{dy}{dx}$. \\[11pt]
\textbf{Sol 1.1.2:} \begin{align*}
    \dfrac{dy}{dx} & = \diff[3x^2 + 5x + 1] \\[11pt]
    & = (3 \cdot 2)x^{2 - 1} + (5 \cdot 1)x^{1 - 1} + 0 \\[11pt]
    & = \boxed{6x + 5} \\
\end{align*}

\textbf{Ex 1.1.3:} $f(x) = x^2 + 4x - 3 + e^x$; find $f'(x)$. \\[11pt]
\textbf{Sol 1.1.3:} \begin{align*}
    f'(x) & = \diff\left[x^2 + 4x - 3 + e^x\right] \\[11pt]
    & = (1 \cdot 2)x^{2 - 1} + (4 \cdot 1)x^{1 - 1} - 0 + e^x \\[11pt]
    & = \boxed{2x + 4 + e^x} \\
\end{align*}

\textbf{Ex 1.1.4:} $y = \sqrt{x^3} + \dfrac{4}{\sqrt{x}} - \sqrt[4]{x^3} + e^4$; find $\dfrac{dy}{dx}$. \\[11pt]
\textbf{Sol 1.1.4:} \begin{align*}
    & y \begin{aligned}[t]
        & = \sqrt{x^3} + \frac{4}{\sqrt{x}} - \sqrt[4]{x^3} + e^4 \\[11pt]
        & = x^{\frac{3}{2}} + 4x^{-\frac{1}{2}} - x^{\frac{3}{4}} + e^4
    \end{aligned} \\[11pt]
    & \frac{dy}{dx} \begin{aligned}[t]
        & = \diff\left[x^{\frac{3}{2}} + 4x^{-\frac{1}{2}} - x^{\frac{3}{4}} + e^4\right] \\[11pt]
        & = \left(1 \cdot \dfrac{3}{2}\right)x^{\frac{3}{2} - 1} + \left(4 \cdot -\dfrac{1}{2}\right)x^{-\frac{1}{2} - 1} - \left(1 \cdot \dfrac{3}{4}\right)x^{\frac{3}{4} - 1} + 0 \\[11pt]
        & =\boxed{\dfrac{3}{2}x^{\frac{1}{2}} - 2x^{-\frac{3}{2}} - \dfrac{3}{4}x^{-\frac{1}{4}}} \\
    \end{aligned}
\end{align*}

Note in Ex 1.1.4 that $e^4$ is a constant. Therefore, its derivative is $0$. \par
As we have seen, when the variable was in the exponent, we use the Exponential Rules. When the variable was in the base, we used the Power Rule. But what if the variable is in both \\[5.5pt] places, such as $\diff\left[(2x - 1)^{x^2}\right]$? It is definitely an exponential problem, but the base is not \\[5.5pt] a constant as the rules above have. The Change of Base Rule allows us to clarify the problem: \begin{align*}
    \diff\left[(2x - 1)^{x^2}\right] = \diff\left[e^{x^2\ln(2x - 1)}\right]
\end{align*} but we will need the Product Rule for this derivative. Therefore, we will save this for later.

\textbf{Ex 1.1.5:} If $y = (x^2 + 1)(x^3 - 4x)$, find $\dfrac{dy}{dx}$. \\[11pt]
\textbf{Sol 1.1.5:} \begin{align*}
    & y \begin{aligned}[t]
        & = (x^2 + 1)(x^3 - 4x) \\[11pt]
        & = x^5 - 4x^3 + x^3 - 4x \\[11pt]
        & = x^5 - 3x^3 - 4x
    \end{aligned} \\[11pt]
    & \dfrac{dy}{dx} \begin{aligned}[t]
        & = \frac{d}{dx}\left[x^5 - 3x^3 - 4x\right] \\[11pt]
        & = \boxed{5x^4 - 9x^2 - 4} \\
    \end{aligned}
\end{align*} 

\textbf{Ex 1.1.6:} If $y = \dfrac{x^2 - 4x + 6}{\sqrt[3]{x}}$, find $\dfrac{dy}{dx}$. \\[11pt]
\textbf{Sol 1.1.6:} \begin{align*}
    & y \begin{aligned}[t]
        & = \dfrac{x^2 - 4x + 6}{\sqrt[3]{x}} \\[11pt]
        & = \dfrac{x^2 - 4x + 6}{x^{\frac{1}{3}}} \\[11pt]
        & = x^{\frac{5}{3}} - 4x^{\frac{2}{3}} + 6x^{-\frac{1}{3}}
    \end{aligned} \\[11pt]
    & \dfrac{dy}{dx} \begin{aligned}[t]
        & = \diff\left[x^{\frac{5}{3}} - 4x^{\frac{2}{3}} + 6x^{-\frac{1}{3}}\right] \\[11pt]
        & = \boxed{\dfrac{5}{3}x^{\frac{2}{3}} - \dfrac{8}{3}x^{-\frac{1}{3}} - 2x^{-\frac{4}{3}}}
    \end{aligned}
\end{align*}

\newpage

\textbf{\large{The Chain Rule}} \par

\textit{Composite Function} $\rightarrow$ A function made of two other functions, one within the other. For example, $y = \sqrt{16x - x^3}$, $y = \sin{x^3}$. $y = \cos^3{x}$, and $y = \left(x^2 + 2x - 5\right)^3$. The general symbol is $f(g(x))$. \par

\textbf{Ex 1.1.7:} Given $f(x) = \cos^{-1} x$, $g(x) = x^2 - 1$, and $h(x) = \sqrt{1 + x^2}$, find a) $f\left(g\left(\sqrt{2}\right)\right)$, b) $h(g(1))$, and c) $f(h(g(1)))$. \\[11pt]
\textbf{Sol 1.1.7:} \begin{align*}
    &\text{(a) } g\left(\sqrt{2}\right) = \left(\sqrt{2}\right)^2 - 1 = 1, \text{ so } f\left(g\left(\sqrt{2}\right)\right) = f(1) = \cos^{-1}(1) = \boxed{0}. \\[11pt]
    &\text{(b) } g(1) = 0, \text{ so } h(g(1)) = h(0) = \sqrt{1 + 0^2} = \boxed{1} \\[11pt]
    &\text{(c) } g(1) = 0 \text{ and } h(g(1)) = h(0) = \sqrt{1 + 0^2} = 1, \text{ so } f(h(g(1))) = \cos^{-1}(1) = \boxed{0}
\end{align*}

So. How do we take the derivative of a composite function? There are two (or more) functions that must be differentiated, but, since one is inside the other, the derivatives cannot be taken at the same time. Just as a radical cannot be distributed over addition, a derivative cannot be distributed concentrically. The composite function is like a matryoshka (Russian doll) that has a doll inside a doll. The derivative is akin to opening them. They cannot both be opened at the same time and, when one is opened, there is an unopened one within. The result is two open dolls adjacent to each other. 

\fbox{\fbox{\begin{minipage}{0.96 \textwidth}
    \vspace{11pt}
    \begin{center}
        The Chain Rule: $\diff \left[f(g(x))\right] = f'(g(x)) \cdot g'(x)$
    \end{center}
    \vspace{11pt}
\end{minipage}}}

If you think of the inside function (the $g(x)$) as equaling $u$, we could write the Chain Rule like this: \begin{align*}
    \diff[f(u)] = \dfrac{df}{du} \cdot \dfrac{du}{dx}
\end{align*}
This is the way that most derivatives are written with the Chain Rule. \par

The Chain Rule is one of the cornerstones of Calculus. It can be embedded within each of the other rules, as seen in the introduction to this chapter. So the Power Rule and Exponential Rules in the last section really should have been stated as: 

\fbox{\fbox{\begin{minipage}{0.96 \textwidth}
    \begin{align*}
        & \text{The Power Rule:} && \text{The Exponential Rules:} \\[11pt]
        & \diff \left[u^n\right] = nu^{n - 1} \cdot \dfrac{du}{dx} && \diff \left[e^u\right] = e^u \cdot \dfrac{du}{dx} \\[11pt]
        & && \diff \left[a^u\right] = \left(a^u \cdot \ln{a}\right) \cdot \dfrac{du}{dx} \\
    \end{align*}
    \begin{center}
        (where $u$ is a function of $x$)
    \end{center}
    \vspace{11pt}
\end{minipage}}}

\textbf{Ex 1.1.8:} $\diff \left[(4x^2 - 2x - 1)^{10}\right]$ \\[11pt]
\textbf{Sol 1.1.8:} \begin{align*}
    & u = 4x^2 - 2x - 1 \text{ and } f(u) = u^{10} \\[11pt]
    & \diff \left[f(u)\right] \begin{aligned}[t]
        & = f'(u) \cdot \dfrac{du}{dx} \\[11pt] 
        & = 10u^9 \cdot \left(8x - 2\right) \\[11pt]
        & = \boxed{10\left(4x^2 - 2x - 1\right)^9(8x - 2)} \\[11pt]
    \end{aligned}
\end{align*}

\textbf{Ex 1.1.9:} $\diff \left[e^{4x^2}\right]$ \\[11pt]
\textbf{Sol 1.1.9:} \begin{align*}
    & u = 4x^2 \\[11pt]
    & \diff \left[e^u\right] \begin{aligned}[t]
        & = e^u \cdot \dfrac{du}{dx} \\[11pt]
        & = e^{4x^2} \cdot 8x \\[11pt]
        & = \boxed{8xe^{4x^2}}
    \end{aligned}
\end{align*}

\textbf{Ex 1.1.10:} If $y = \sqrt{16 - x^3}$, find $\dfrac{dy}{dx}$. \\[11pt]
\textbf{Sol 1.1.10:} \begin{align*}
    & u = 16 - x^3 \text{ and } f(u) = \sqrt{u} \\[11pt]
    & \dfrac{dy}{dx} \begin{aligned}[t]
        & = \diff \left[f(u)\right] = f'(u) \cdot \dfrac{du}{dx} \\[11pt] 
        & = \dfrac{1}{2\sqrt{u}} \cdot -3x^2 \\[11pt]
        & = \boxed{\dfrac{-3x^2}{2\sqrt{16 - x^3}}}
    \end{aligned}
\end{align*} 

\textbf{Ex 1.1.11:} $\diff \left[\sqrt{\left(x^2 + 1\right)^5 + 7}\right]$ \\[11pt]
\textbf{Sol 1.1.11:}\begin{align*}
    & u = x^2 + 1, \; g(u) = u^5 + 7, \text{ and } f(g(u)) = \sqrt{g(u)} \\[11pt]
    & \diff \left[f(g(u))\right] \begin{aligned}
        & = f'(g(u)) \cdot g'(u) \cdot \dfrac{du}{dx} \\[11pt]
        & = \dfrac{1}{2\sqrt{g\left(u\right)}} \cdot 5u^4 \cdot 2x \\[11pt]
        & = \dfrac{5\left(x^2 + 1\right)(2x)}{2\sqrt{\left(x^2 + 1\right)^5 + 7}} = \boxed{\dfrac{5x\left(x^2 + 1\right)}{\sqrt{\left(x^2 + 1\right)^5 + 7}}}
    \end{aligned}
\end{align*}

\newpage

\textbf{\large{1.1 Free Response Homework}} \par

Find the derivatives of the given functions. Simplify where possible. \par

\twoquestion{\text{1. } f(x) = x^2 + 3x - 4}{\text{2. } f(t) = \dfrac{1}{4}\left(t^4 + 8\right)} \\[11pt]
\twoquestion{\text{3. } y = x^{-\frac{2}{3}}}{\text{4. } y = 5e^x + 3} \\[11pt]
\twoquestion{\text{5. } v(r) = \dfrac{4}{3} \pi r^3}{\text{6. } g(x) = x^2 + \dfrac{1}{x^2}} \\[11pt]
\twoquestion{\text{7. } y = \dfrac{x^2 + 4x + 3}{\sqrt x}}{\text{8. } u = \sqrt[3]{t^2} + 2\sqrt{t^3}} \\[11pt]
\twoquestion{\text{9. } z = \dfrac{A}{y^{10}} + Be^y}{\text{10. } y = e^{x + 1} + 1} \\[11pt]

Complete the following. \par

\twoquestion{\text{11. } \diff \left[x^7 - 4\sqrt[8]{x^7} + 7^x - \dfrac{1}{\sqrt[7]{x^4}} + \dfrac{1}{5x}\right]}{\text{12. } \diff \left[x^6 - 3\sqrt[6]{x^7} + 5^x - \dfrac{1}{\sqrt[3]{x^5}} + \dfrac{1}{8x} \right]} \\[11pt]
\twoquestion{\text{13. } \diff \left[x^4 - 14\sqrt[7]{x^9}\right] + 8^x - \dfrac{1}{\sqrt[3]{x^7}} + \dfrac{1}{8x}}{\text{14. } \diff \left[(x - 1)\sqrt{x}\right]} \\[11pt]
\twoquestion{\text{15. } \dfrac{d}{dz} \left[\left(z^2 - 4\right)\sqrt{z^3}\right]}{\text{16. } \diff \left[\left(x^2 - 4x + 3\right)\sqrt{x^5}\right]} \\[11pt]
\twoquestion{\text{17. } \dfrac{d}{dt} \left[\left(4t^2 + 1\right)\left(3t^3 + 7\right)\right]}{\text{18. } \diff \left[\left(x^3 + 4x - \pi\right)^{-7}\right]} \\[11pt]
\twoquestion{\text{19. } \diff \left[\sqrt{3x^2 - 4x + 9}\right]}{\text{20. } \diff \left[\sqrt[7]{x^3 - 2x}\right]} \\[11pt]
\twoquestion{\text{21. } \dfrac{d}{dy} \left[\dfrac{4y^3 - 2y^2 - 5y}{\sqrt y}\right]}{\text{22. } \dfrac{d}{dv} \left[\dfrac{v^2 - 4v + 7} {2\sqrt v}\right]} \\[11pt]
\twoquestion{\text{23. } \dfrac{d}{dw}\left[\dfrac{7w^2 - 4w + 1}{5w^3}\right]}{\text{24. } \dfrac{d}{dw}\left[\dfrac{5w^2 - 3w - 4}{7w^2}\right]} \\[11pt]
\twoquestion{\text{25. } f(x) = \sqrt[4]{1 + 2x + x^3} \text{, find } f'(x)}{\text{26. } f(x) = \sqrt[5]{\left(\dfrac{1}{x} + 2x + e^x\right)^3} \text{, find } f'(x)} \\[11pt]
\twoquestion{\text{27. } f(x) = \left(x^3 + 2x\right)^{37}\text{, find } f'(x)}{\text{28. } f(x) = 3x^5 - 5x^3 + 3 \text{, find } f'(x)} \\[11pt]
\twoquestion{\text{29. } g(2) = 3, \; g'(2) = -4, \; f(x) = e^{g(x)} \text{, find } f'(2)}{\text{30. } y = e^{\sqrt x} \text{, find } \dfrac{dy}{dx}} \\[11pt]
\twoquestion{\text{31. } f(x) = \sqrt{4 - \dfrac{4}{9}x^2} \text{, find } f'\left(\sqrt 5\right)}{\text{32. } f(x) = e^{\sqrt{9 - x^2}} \text{, find } f'(x)} \\[11pt]
\twoquestion{\text{33. } v(t) = \sqrt{\left(\dfrac{E(t)}{3} + 3t\right)^{\frac{3}{7}} - 4}, \text{, find } v'(t)}{\text{34. } v(t) = \sqrt[3]{\left(\dfrac{C(t)}{7} + 4t^2\right)^{\frac{5}{7}} - 1} \text{, find } v'(t)} \\[11pt]

\textbf{\large{1.1 Multiple Choice Homework}} \par

\begin{questions}
    \question If $f(x) = x^\frac{3}{2}$, then $f'(4) =$ \\
    
    \begin{oneparchoices}
        \choice $-6$ 
        \choice $-3$ 
        \choice $3$
        \choice $6$
        \choice $8$
    \end{oneparchoices} \par \horizontalline

    \question The derivative of $\sqrt x - \dfrac{1}{x\sqrt[3]{x}}$ \\

    \begin{oneparchoices}
        \choice $\dfrac{1}{2}x^{-\frac{1}{2}} - x^{-\frac{4}{3}}$
        \choice $\dfrac{1}{2}x^{-\frac{1}{2}} - \dfrac{4}{3}x^{-\frac{7}{3}}$ 
        \choice $\dfrac{1}{2}x^{-\frac{1}{2}} - \dfrac{4}{3}x^{-\frac{1}{3}}$ \\[11pt]
        \makebox[0.2 \textwidth] \choice $-\dfrac{1}{2}x^{-\frac{1}{2}} - \dfrac{4}{3}x^{-\frac{7}{3}}$ 
        \makebox[0.25 \textwidth] \choice $-\dfrac{1}{2}x^{-\frac{1}{2}} - \dfrac{4}{3}x^{-\frac{1}{3}}$
    \end{oneparchoices} \par \horizontalline

    \question Given $f(x) = \dfrac{1}{2x} + \dfrac{1}{x^2}$, find $f'(x)$ \\

    \begin{oneparchoices}
        \choice $-\dfrac{1}{2x^2} - \dfrac{2}{x^3}$ 
        \choice $-\dfrac{2}{x^2} - \dfrac{2}{x^3}$
        \choice $\dfrac{2}{x^2} - \dfrac{2}{x^3}$ \\
        \makebox[0.23 \textwidth] \choice $-\dfrac{1}{2x^2} + \dfrac{2}{x^3}$
        \makebox[0.29 \textwidth] \choice $\dfrac{1}{2x^2} - \dfrac{2}{x^3}$
    \end{oneparchoices} \par \horizontalline

    \question If $f(x) = e^{5x^2} + x^4$, then $f'(1)=$ \\

    \begin{oneparchoices}
        \choice $e^5 + 1$
        \choice $5e^4 + 4$ 
        \choice $5e^5 + 1$
        \choice $10e + 4$
        \choice $10e^5 + 4$
    \end{oneparchoices} \par \horizontalline

    \question If $h$ is the function defined by $h(x) = e^{5x} + x + 3$, then $h'(0) $ is \\

    \begin{oneparchoices}
        \choice 2
        \choice 4
        \choice 5
        \choice 6
        \choice 8
    \end{oneparchoices} \par \horizontalline

    \question If $y = \left(x^4 + 4\right)^2$, then $\dfrac{dy}{dx} = $ \\

    \begin{oneparchoices}
        \choice $2\left(x^4 + 4\right)$
        \choice $\left(4x^3\right)^2$
        \choice $2\left(4x^3 + 4\right)$ \\[11pt]
        \makebox[0.23 \textwidth] \choice $4x^3\left(x^4 + 4\right)$
        \makebox[0.25 \textwidth] \choice $8x^3\left(x^4 + 4\right)$
    \end{oneparchoices} \par \horizontalline

    \question If $h(x) = \left[f(x)\right]^2 g(x)$ and $g(x) = 3$, then $h'(x) = $ \\

    \begin{oneparchoices}
        \choice $2f'(x)g'(x)$
        \choice $6f'(x)f(x)$
        \choice $g'(x)\left[f(x)\right]^2 + 2f(x)f'(x)g(x)$ \\[11pt]
        \makebox[0.23 \textwidth] \choice $2f'(x)g(x) + g'(x)\left[f(x)\right]^2$
        \makebox[0.25 \textwidth] \choice $0$
    \end{oneparchoices} \par \horizontalline

    \question Which of the following statements must be true?
    \begin{align*}
        & \text{I. } \diff \left[\sqrt{e^x + 3}\right] = \dfrac{e^x}{2\sqrt{e^x + 3}} \\[11pt]
        & \text{II. } \diff\left[5^{3x^2}\right] = 6x\ln (5)\left(5^{3x^2}\right) \\[11pt]
        & \text{III. } \diff\left[6x^3 - \pi + \sqrt[3]{x^8} - \dfrac{2}{x^3}\right] = 18x^2 + \dfrac{8}{3}\sqrt[3]{x^5} + \dfrac{6}{x^4}
    \end{align*}

    \begin{oneparchoices}
        \choice I only 
        \choice II only 
        \choice I and III only \\[11pt]
        \makebox[0.2 \textwidth] \choice I and III only 
        \makebox[0.25 \textwidth] \choice I, II, and III
    \end{oneparchoices} \par \horizontalline
\end{questions}