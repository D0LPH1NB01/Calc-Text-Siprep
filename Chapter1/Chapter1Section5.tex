\textbf{\underline{\large{1.5: Intro to AP: Basic Derivatives Numerically and Graphically}}} \par

Traditionally, calculus was an algebraically heavy subject. One of the philosophical changes that the CollegeBoard made in the 1990s was to emphasize that calculus should be understood in a variety of modes. As they state in their enduring understanding: \begin{center}
    \parbox{0.8\textwidth}{\textit{
        ``Students should be able to work with functions represented in a veriety of ways: graphical, numerical, analytical or verbal.  They should understand the connections among these representations.''}
    }
\end{center} \par

Later, they added that students should be able to verbalize their understanding and be able to communicate that understanding through proper writing. We will consider this later as we consider more context-oriented problems.